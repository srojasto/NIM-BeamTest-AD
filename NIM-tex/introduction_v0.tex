\section{Introduction.}

ALICE (A Large Ion Collider Experiment) \cite{AlicePerformance} is designed to study strongly
interacting matter at the highest energy densities reached so far in the laboratory, using
proton-proton, proton-nucleus and 
nucleus-nucleus collisions at the CERN LHC. ALICE started operation in 2009, in Run 1 of the LHC,
during which studies of diffractive processes in proton-proton collisions were initiated
\cite{Villalobos, Evdokimov}. Such studies are based on the ability to define gaps in the
pseudorapidity distributions of particles produced in the collisions. 
It is in this context that the AD detector was proposed, in order to increase the pseudorapidity
coverage of ALICE. 
It was installed around the beam pipe in the forward region; ADC located at 19.5 m from the
interaction point (IP) inside the LHC tunnel and ADA situated on the other side at 17 m from the
IP of ALICE, behind the compensator magnet on the A-Side \cite{ADNote, AD-Villatoro}.
% 
Each sub-detector or station is an assembly of two layers segmented in four pads. 
The pads are made of plastic scintillator coupled at %
two wavelength shifter bars (at the sides of the pad), they are separated by 0.4 millimeters of
air to maximize the light transmission to the WLS bars and collect the light produced inside
scintillators.
%
To transport the light of each pad a bundle of 192 fibres (96 for each WLS bar) are coupled to one
PMT (Hamamatsu R5946) to convert the light into an electronic signal. A detailed description can
be seen in \cite{ADNote}.

Two AD modules identical to those installed in the experiment were studied in the T10 Proton
Synchroton (PS) test beam, one ADC and  ADA modules, labeled AD1 and AD2 respectively, in order to
determine the homogeneity along the pads, sensitivity to a MIP particle and measure the relative
response for charge and time of protons, pions and muons at different momentums (1, 1.5,6 and 6 GeV/c). 

