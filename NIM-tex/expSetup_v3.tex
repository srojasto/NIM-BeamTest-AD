\section{Experimental Setup.} \label{cap:ExpSetup}

An ADA and ADC pads were tested, these pads are identical to the modules in the experiment, except for the fibres lengths (47 cm in the tested modules). Each bunch of fibres was coupled to a PMT and for the readout we used an electronic system identical to the one installed in ALICE \cite{Zoccarato}. For the beam test the pads were labeled as AD1 and AD2 for ADA and ADC respectively. In addition two scintillator hodoscopes and two Cherenkov  radiators were used during the beam test. %The hodoscopes were also used to trigger cosmics.
\\
%\subsection{T10 beam line.}
To study the performance of AD under controlled conditions we use the facilities from T10 \cite{t10BeamArea} which delivers secondary particles, mainly pions ($\pi^{+}$) and protons ($p^{+}$) produced in the Proton Synchrotron (PS) machine \cite{Simon:PS,psBeam}. It is possible to adjust the parameters from the beam such like: momentum of the particles, beam size spot and beam focus. For the beam test we use four different momentum values: 1.0, 1.5, 2.0 and 6 GeV/c, with a resolution of 1.3\%.

%\subsection{Experimental setup configurations.}
The setup consists of two different configurations, one used to measure cosmic rays and another for particles from T10 beam. In both cases the nominal voltages were fixed to 1650 and 1500 Volts for AD1 and AD2 respectively.
The first setup was placed in order to obtain the MIP position with cosmic rays \cite{cosmicsAD}, two scintillator hodoscopes, labeled as \textit{Black start} and \textit{Black end}, were placed above and below of AD1 and AD2. The setup for the beam test in T10 is shown in Figure \ref{figure:BeamSetup}. The arrangement is labeled in the beam direction, indicated by the red arrow, and corresponds to: Two hodoscopes, Black-start and Black-end separated by a distance of 1221 cm and two Cherenkov radiators (provided by T0 detector group) labelled as T0-start and T0-end.
%
To readout the signals from all detectors we used a standalone system identical to the Data Acquisition installed in the experiment \cite{ADNote}. 
%\input{NIM-tex/redoutElectronic1.tex}
%
%\subsubsection{Trigger.}\label{subsection:trigger}   
Is important to point out that the data acquisition was triggered by define a logical \texttt{AND} either by the hodoscopes or T0 detectors. The signals passes for a preamplifier that split the signal in to two, one amplified by ten and another by one, that are used by the Front End Electronics (FEE) to measure time and charge respectively.
%When the pulse amplified by a factor of 10 crosses the threshold level it generates a leading edge  trigger, in our case the threshold level was fixed to $-40$ mV. 
%\input{NIM-tex/trigger.tex}
\begin{figure}[ht!]
	\begin{center}
		\includegraphics[scale=0.45]{images/BeamTest-Setup_v4.pdf}
		\caption{Beam test setup scheme installed at T10 beam line and its separation distances in centimeters. 
			The red arrow show the direction of the beam. 
		}
		\label{figure:BeamSetup}
	\end{center}
\end{figure}
%\input{NIM-tex/setup_conf.tex}